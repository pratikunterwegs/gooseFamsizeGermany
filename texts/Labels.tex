\documentclass[]{article}
\usepackage{lmodern}
\usepackage{amssymb,amsmath}
\usepackage{ifxetex,ifluatex}
\usepackage{fixltx2e} % provides \textsubscript
\ifnum 0\ifxetex 1\fi\ifluatex 1\fi=0 % if pdftex
  \usepackage[T1]{fontenc}
  \usepackage[utf8]{inputenc}
\else % if luatex or xelatex
  \ifxetex
    \usepackage{mathspec}
  \else
    \usepackage{fontspec}
  \fi
  \defaultfontfeatures{Ligatures=TeX,Scale=MatchLowercase}
\fi
% use upquote if available, for straight quotes in verbatim environments
\IfFileExists{upquote.sty}{\usepackage{upquote}}{}
% use microtype if available
\IfFileExists{microtype.sty}{%
\usepackage{microtype}
\UseMicrotypeSet[protrusion]{basicmath} % disable protrusion for tt fonts
}{}
\usepackage[unicode=true]{hyperref}
\hypersetup{
            pdfborder={0 0 0},
            breaklinks=true}
\urlstyle{same}  % don't use monospace font for urls
\IfFileExists{parskip.sty}{%
\usepackage{parskip}
}{% else
\setlength{\parindent}{0pt}
\setlength{\parskip}{6pt plus 2pt minus 1pt}
}
\setlength{\emergencystretch}{3em}  % prevent overfull lines
\providecommand{\tightlist}{%
  \setlength{\itemsep}{0pt}\setlength{\parskip}{0pt}}
\setcounter{secnumdepth}{0}
% Redefines (sub)paragraphs to behave more like sections
\ifx\paragraph\undefined\else
\let\oldparagraph\paragraph
\renewcommand{\paragraph}[1]{\oldparagraph{#1}\mbox{}}
\fi
\ifx\subparagraph\undefined\else
\let\oldsubparagraph\subparagraph
\renewcommand{\subparagraph}[1]{\oldsubparagraph{#1}\mbox{}}
\fi

\date{}

\begin{document}

\emph{a.} Wintering grounds of greater white-fronted geese \emph{Anser
a. albifrons} in the Netherlands and northern Germany, with 65 sites
(dots) where 51,037 successful families in 1,884 flocks were recorded.
21 splits (diamonds) were observed in 13 GPS tracked families. Shaded
area bounds 10,635 observations of marked geese. Data were collected
from 2000 - 2017. \emph{b.} Breeding grounds (ellipse) with Kolguyev
Island (dot) and rough migration route (arrow) to wintering area
(rectangle).

\section{Fig.2 Distance \textasciitilde{} family
size}\label{fig.2-distance-family-size}

GLMM fits (lines), and mean distance of wintering sites from Kolguyev
Island (symbols) per number of juveniles in a family. Data and fit for
data collected \textless{} 60 days after arrival are shown in red; data
and fit for records \textgreater{} 60 days after arrival are in blue.
Triangles \& dotted lines represent data from marked geese (dataset
\emph{C}), circles and solid lines family counts (dataset \emph{B}).

\section{Fig.3 Family \textasciitilde{} time}\label{fig.3-family-time}

GLMM fits (lines) and mean number of juveniles per family on each day since goose autumn arrival pooled across years (dots). Successful families in flocks (dataset \emph{B}) are shown in red, and families of
marked geese (dataset \emph{C}) are shown in blue. Arrows show development of size of 9 GPS tracked families that underwent splits.

\section{Fig.4 Family size \textasciitilde{}
predation}\label{fig.4-family-size-predation}

GLMM partial fit (lines) and mean number of juveniles per family at each
unique level of pooled summer predation index (symbols) using two
datasets: blue, all families of marked geese (dataset \emph{C}); red,
successful families cointed in flocks (dataset \emph{B}); black,
successful families only of marked geese (subset of \emph{C}).

\section{Fig.5 N fams \textasciitilde{} flock
size}\label{fig.5-n-fams-flock-size}

GAMM partial fit (line) and mean number of successful families in
white-fronted goose flocks of each unique size (circles). 95\%
confidence interval is shaded grey.

\section{Fig.6 N fams \textasciitilde{} time}\label{fig.6-n-fams-time}

GAMM partial fit (line) and mean number of successful families in
white-fronted goose flocks on each winter day, pooled across all winters
(circles). 95\% confidence interval is shaded grey.

\section{Fig.7 N fams \textasciitilde{} dist}\label{fig.7-n-fams-dist}

GAMM fit (line) and mean number of successful families in
white-fronted goose flocks at each site (circles, n = 49) as a function
of its distance from the Kolguyev Island. 95\% confidence interval is
shaded grey.

\section{Fig.8 Flocksize \textasciitilde{}
distance}\label{fig.8-flocksize-distance}

GLMM partial fit (line) and mean size of flocks at each site (circles, n
= 111) as a function of its distance from Kolguyev Island.

\section{Fig.9 J\% \textasciitilde{} time}\label{fig.9-j-time}

GAMM partial fit (line) and mean proportion of first-winter juveniles in
white-fronted goose flocks on each winter day, pooled across all years
(circles). Note that days since arrival was modelled as a smoothed
covariate using thin plate splines, and 4 knots, with the smooth forced
through 0. Dashed lines bound the 95\% confidence interval.

\section{Fig.10 p(split) \textasciitilde{} days, total flights, total
distance, family
size}\label{fig.10-psplit-days-total-flights-total-distance-family-size}

GAMM partial fits (lines) for (a) days since arrival, (b) cumulative
number of flights over winter, (c) number of juveniles, and (d)
cumulative number of displacements of more than 1000 km.

\end{document}
